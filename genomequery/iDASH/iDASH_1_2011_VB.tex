\documentclass[10pt]{article}
%\documentclass[10pt,fullpage]{article}
%\usepackage[T1]{fontenc}
%\usepackage[latin9]{inputenc}
%\bibliographystyle{plain}
\usepackage{helvet}
%\usepackage{times}
\usepackage{epsfig}
\usepackage[small,compact]{titlesec}
\usepackage[reqno]{amsmath}
\usepackage{color}
\usepackage{fancybox}
\bibliographystyle{plain}
\usepackage[super]{natbib}
\setlength{\evensidemargin}{0in}
%\setlength{\evensidemargin}{-0.15in}
%\setlength{\oddsidemargin}{-0.15in}
%\setlength{\oddsidemargin}{0in}
%\setlength{\marginparwidth}{0.0in}
%\setlength{\textwidth}{6.50in}
%\setlength{\textheight}{9.0in}
%\setlength{\topmargin}{0.55in}
%\setlength{\topmargin}{0in}
%\setlength{\headheight}{0in}
%\setlength{\headsep}{0in}
%\setlength{\columnsep}{0.20in}
\pagestyle{plain}
\makeatletter
%%%%%%%%%%%%%%%%%%%%%%%%%%%%%% User specified LaTeX commands.



\usepackage{epsfig,longtable}
%\usepackage{fullpage,doublespace}
%\usepackage{psfrag}
%\usepackage{genres}
%\usepackage{times}
%\usepackage{latexsym}
%\usepackage{fancybox,subfigure}
\usepackage[normalem]{ulem}
%\usepackage{normalem}
%\bibliographystyle{plain}
%\usepackage{amsmath}
%\usepackage{mathrsfs}
%\usepackage{algorithmic}
%\usepackage{tweaklist}

\newcommand{\captionfonts}{\small}
\usepackage{float}
\floatstyle{plain}
\newfloat{supplementalfigure}{thp}{sup}
\floatname{supplementalfigure}{Figure S\hspace{-3pt}}
\newfloat{supplementaltable}{thp}{sup}
\floatname{supplementaltable}{Table S\hspace{-3pt}}

%%%
\newtheorem{problem}{Problem}

%\makeatletter
%\def\@cite#1#2{$^{\mbox{\tiny #1\if@tempswa , #2\fi}}$}
%\makeatother

\makeatletter  % Allow the use of @ in command names
\long\def\@makecaption#1#2{%
  \vskip\abovecaptionskip
  \sbox\@tempboxa{{\captionfonts #1: #2}}%
  \ifdim \wd\@tempboxa >\hsize
    {\captionfonts #1: #2\par}
  \else
    \hbox to\hsize{\hfil\box\@tempboxa\hfil}%
  \fi
  \vskip\belowcaptionskip}

%\def\aligntop#1{\setbox\@tempboxa \hbox{#1}%
%                \@tempdima=\ht\@tempboxa%
%                \advance\@tempdima-\ht\strutbox%
%                \leavevmode\lower\@tempdima\box\@tempboxa}

\makeatother   % Cancel the effect of \makeatletter

\setlength{\topmargin}{-0.5in}
\usepackage{latexsym}
\setlength{\columnsep}{0.5cm} 
\setlength{\oddsidemargin}{-0cm}
\setlength{\evensidemargin}{-0cm} 
\setlength{\textwidth}{6.8in}
\setlength{\textheight}{8.7in}

\newcommand{\old}[1]{}
\newcommand{\match}{\stackrel{M}{=}}
\newcommand{\ncite}[1]{$^{\mbox{\tiny \cite{#1}}}$}
%\newcommand{\nncite}[1]{\cite{#1}}
\newcommand{\frags}{{\cal F}}
\newcommand{\snips}{{\cal S}}
\newcommand{\A}{{\tt A}}
\newcommand{\B}{{\tt B}}
\newcommand{\gap}{{\tt -}}
\newcommand{\ideas}{\vskip 0.6cm {\bf IDEAS:\ }}
\newcommand{\motivation}{\vskip 0.6cm {\bf MOTIVATION:\ }}
\newcommand{\mcost}[2]{#1 #2}
\newcommand{\ali}{$\mbox{ }$\hspace{0.1in}}
\newcommand{\acomment}[1]{\hspace{1in}\#{\em #1}}
\newcommand{\beqn}{\begin{equation}}
\newcommand{\eeqn}{\end{equation}}
\newcommand{\comment}[1]{******* {\em #1} *******}
%\newcommand{\LtoN}[1]{\parallel #1 \parallel_{2}}
\newcommand{\LtoN}[1]{\left\Vert {#1} \right\Vert}
\newcommand{\Ntr}[1]{\frac{\vecbf{#1}-P_{#1}}{\sqrt{P_{#1}P_{\bar{#1}}}}}

\newcommand{\BIN}[1]{\left\langle{#1}\right\rangle}
\newcommand{\ABS}[1]{\left|{#1}\right|}
\newcommand{\FLOOR}[1]{\left\lfloor{#1}\right\rfloor}
\newcommand{\CEIL}[1]{\left\lceil{#1}\right\rceil}
%\newcommand{\SET}[1]{\left\{{#1}\right\}}
\newcommand{\SET}[1]{\{{#1}\}}
\newcommand{\Rapid}{{\sc Rapid}}
\newcommand{\subbox}[1]{\mbox{\footnotesize #1}}

\newcommand{\captionsize}{\footnotesize}
\newcommand{\mecca}{HapCUT$\:$}
%\newcommand{\vecbf}[1]{{\bf #1}}
\let\vecbf\boldsymbol
%\newcommand\vecbf[1]{\boldsymbol{\vec #1}}
%%%%%%%%%%%%%% Figure within a box
\newenvironment{boxfig}[1]{\fbox{\begin{minipage}{\linewidth}
                        \vspace{1em}
                        \makebox[0.025\linewidth]{}
                        \begin{minipage}{0.95\linewidth}
                        #1
\end{minipage}
                        \end{minipage}}}

\newcommand{\proc}[1]{\ensuremath{\mbox{\sc #1}}}

\newcommand{\MST}{\ensuremath{\mathit{MST}}}
\newcommand{\dist}{\ensuremath{\mathrm{dist}}}
\newcommand{\TG}[2]{\ensuremath{\mathit{#1}^{(#2)}}}
\newcommand{\CC}{\ensuremath{\mathcal{CC}}}
\newcommand{\psubs}{\stackrel{\subset}{+}}
\newcommand{\rs}{\ensuremath{\mathit{R_s}}}
\newcommand{\MEC}{\ensuremath{\mathit{MEC}}}
\newcommand{\Prob}{\ensuremath{\mbox{Pr}}}
\newcommand{\Exp}{\ensuremath{\mbox{E}}}

%%%%%%%%%%%%%%%%%%%%%%%%%%%%%  THEOREM-LIKE ENVIRONMENTS

\newtheorem{THEOREM}{{\bf  Theorem}}
\newenvironment{theorem}{\begin{THEOREM} \hspace{-.85em}  {\bf :} }%
                        {\end{THEOREM}}
\newtheorem{LEMMA}[THEOREM]{Lemma}
\newenvironment{lemma}{\begin{LEMMA} \hspace{-.85em} {\bf :} }%
                      {\end{LEMMA}}
\newtheorem{COROLLARY}[THEOREM]{Corollary}
\newenvironment{corollary}{\begin{COROLLARY} \hspace{-.85em} {\bf :} }%
                          {\end{COROLLARY}}
\newtheorem{PROPOSITION}[THEOREM]{Proposition}
\newenvironment{proposition}{\begin{PROPOSITION} \hspace{-.85em} {\bf :} }%
                            {\end{PROPOSITION}}
\newtheorem{CLAIM}[THEOREM]{Claim}
\newenvironment{claim}{\begin{CLAIM} \hspace{-.85em} {\bf :} }%
                      {\end{CLAIM}}
\newtheorem{OBSERVATION}[THEOREM]{Observation}
\newenvironment{Observation}{\begin{OBSERVATION} \hspace{-.85em} {\bf :} }%
                      {\end{OBSERVATION}}
\newtheorem{DEFINITION}{Definition}
\newenvironment{definition}{\begin{DEFINITION} \hspace{-.85em} {\bf :} }%
                           {\end{DEFINITION}}
\newcommand{\QED}{\hfill$\clubsuit$ \vskip 0.1cm}
\newenvironment{proof}{\noindent {\bf Proof:} \hspace{.677em}}{\QED}


\begin{document}

\title{\vspace{-1in} Layering for Genomics: iDASH Report January}
\author{Vineet Bafna\thanks{CSE 4218, Univ. California, San Diego. \{vbafna,gvarghese\}@cs.ucsd.edu} \and George Varghese$^*$}
\date{}
\maketitle
\section{January, 2011 update}

\begin{itemize}
\item We propose continued collaborations with School of Medicine on
  the development of tools for analyzing genomic data. 
\emph{
  \begin{itemize}
  \item In collaboartion with Prof. Gabriel Haddad, we published a
    study of hypoxia tolerance in Drososphila (PNAS ePUB January 24,
    2011), and are negotiating with BGI for sequencing of human
    highlander populations\cite{Zhou2011}.
  \item We published a novel approach to haplotyping using Probe
    sequencing technologies from Pacific Biosciences\cite{Lo2011}.
  \end{itemize}
}


\item We propose to develop and publish a layered abstraction of
  specific software modules that process, map, compress and query the donor
  data for cataloging variations, with precise descriptions of the
  interfaces between modules.  We hope our layering will provide a context for
  different software generated by the larger
  research community, and help with Life Tech's goal of standardizing
  tools for genome sequences.
\emph{
  \begin{itemize}
  \item In progress. We are starting by implementing specific queries. 
  \end{itemize}
}

\item We propose a prototype implementation of two specific software
  layers. First, we will develop a compression layer, extending the
  ideas presented in Kozanitis et al., 2010. The second is an
  \emph{evidence layer (EL)}. The EL is a collection of APIs that
  efficiently retrieve \emph{all} raw data relevant to inferring
  specific variations.  Note that our proposal separates out an
  evidence layer (EL) from a separate inference layer (IL), two layers
  that are commonly intertwined in existing software.
\emph{
  \begin{itemize}
  \item We have developed a query for interrogating large structural deletions that runs on 40X human genome data in less than 60 minutes to identify all possible deletions.
  \end{itemize}
}
\end{itemize}


Previous publications\citep{bhatia2010, brinza2010, Dost2010,
  conf:kozanitis2010, Harismendy2010, Levy2007, Zhou10}, and helps
guide the way genomes are being queried and analyzed.

As a first step towards efficient querying of genomic data, we have
developed a tool for genome compression\cite{conf:kozanitis2010}. We
have initated collaborations with Life Technologies, and Illumina,
which will provide us with valuable data and access to end-users.


\bibliographystyle{plain}
\bibliography{idash}
\end{document}